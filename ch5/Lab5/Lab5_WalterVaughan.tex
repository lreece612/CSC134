% Lab 5 writeup
% written by Walter B. Vaughan

\documentclass[11pt]{article}
\author{Walter B. Vaughan\\
        \small CSC 134 -- Section 200 -- Fall 2014\\
        \small Catawba Valley Community College}
\title{Lab 5}
\date{\vspace{-5ex}}


\usepackage{listings}
\usepackage[usenames,dvipsnames]{color}
\usepackage[margin=.5in]{geometry}
\usepackage{indentfirst}


\lstset{frame=tb,
	language=C++,
	columns=flexible,
	basicstyle={\small\ttfamily},
	numbers=left,
	numberstyle=\color{Gray},
	keywordstyle=\color{BlueViolet},
	commentstyle=\color{Gray},
	stringstyle=\color{OliveGreen},
	breaklines=true,
	breakatwhitespace=true,
	showstringspaces=false,
	tabsize=4
}

\begin{document}

\maketitle

\section*{Pre-Lab Questions}
\begin{enumerate}
	\item A block of code that repeats forever is called \emph{an infinite loop}.
	\item To keep track of the number of times a particular loop is repeated, one
can use a \emph{counter}.
	\item An event controlled loop that is always executed at least once is the \emph{post-test loop}.
	\item An event controlled loop that is not guaranteed to execute at least once is
the \emph{pre-test loop}.
	\item In the conditional \lstinline{if(++number < 9)}, the comparison \lstinline{number < 9} is made \emph{second} and \lstinline{number} is incremented \emph{first}.
	\item In the conditional \lstinline{if(number++ < 9)}, the comparison \lstinline{number < 9} is made \emph{first} and \lstinline{number} is incremented \emph{second}.
	\item A loop within a loop is called a \emph{nested loop}.
	\item List the preprocessor directive that is used to allow data and output files to
be used in the program. \lstinline{#include <fstream>}
	\item To write out the first 12 positive integers and their cubes, one should use
a \emph{for} loop.
	\item A \emph{sentinel} value is used to indicate the end of a list of values. It can be used to control a \lstinline{while} loop.
	\item In a nested loop the \emph{inner} loop goes through all of its iterations for each iteration of the \emph{outer} loop.
\end{enumerate}


\section*{\textsc{Lab 5.1} --- Working with the \texttt{while} Loop}

	\textbf{Exercise 1:} The inital version of the program \texttt{while.cpp} is not user-friendly because from the outside, there is no indication of how to keep the program from looping endlessly. There should be a notice that typing the character \lstinline{`x`} will end the program.
	
	\textbf{Exercise 3:} Changing the \lstinline{while} loop to a \lstinline{do-while} loop yields no change, because the loop is already guaranteed to execute at least once thanks to the assignment \lstinline{char letter = 'a';}.
	
	\textbf{Exercise 5:} Running the program after adding the required code yields the expected results. Even inputting \lstinline{0} for one or more months gives correct results. Entering \lstinline{-1} for the first month exits the program and correctly displays the message \lstinline{"No data has been entered"}. The only numerical data that should not be input is negative rainfall amounts, which while syntactically correct, are logically inconsistent with the problem.
	
	\textbf{Exercise 6:} The purpose of the \lstinline{if (month == 1)} section is to avoid stating the total rainfall over 0 months, under the assumption that the first number input was \lstinline{-1} and \lstinline{month} was never incremented.


\section*{\textsc{Lab 5.2} --- Working with the \texttt{do-while} Loop}

	\textbf{Exercise 1} was performed successfully, and the record of my testing is as follows:
	
	\small\begin{verbatim}
wbv@mbp-arch > make dowhile.o
g++ -Wall -pedantic dowhile.cpp -o dowhile.o
wbv@mbp-arch > ./dowhile.o 


Hot Beverage Menu

A: Coffee         $1.00
B: Tea            $ .75
C: Hot Chocolate  $1.25
D: Cappuccino     $2.50


Enter the beverage A,B,C, or D you desire
Enter E to exit the program

a
How many cups would you like?
1
The total cost is $ 1.00


Hot Beverage Menu

A: Coffee         $1.00
B: Tea            $ .75
C: Hot Chocolate  $1.25
D: Cappuccino     $2.50


Enter the beverage A,B,C, or D you desire
Enter E to exit the program

B
How many cups would you like?
4
The total cost is $ 3.00


Hot Beverage Menu

A: Coffee         $1.00
B: Tea            $ .75
C: Hot Chocolate  $1.25
D: Cappuccino     $2.50


Enter the beverage A,B,C, or D you desire
Enter E to exit the program

C
How many cups would you like?
99
The total cost is $ 123.75


Hot Beverage Menu

A: Coffee         $1.00
B: Tea            $ .75
C: Hot Chocolate  $1.25
D: Cappuccino     $2.50


Enter the beverage A,B,C, or D you desire
Enter E to exit the program

d
How many cups would you like?
3
The total cost is $ 7.50


Hot Beverage Menu

A: Coffee         $1.00
B: Tea            $ .75
C: Hot Chocolate  $1.25
D: Cappuccino     $2.50


Enter the beverage A,B,C, or D you desire
Enter E to exit the program

g
Selection 'g' is invalid.
 Try again please


Hot Beverage Menu

A: Coffee         $1.00
B: Tea            $ .75
C: Hot Chocolate  $1.25
D: Cappuccino     $2.50


Enter the beverage A,B,C, or D you desire
Enter E to exit the program

G
Selection 'G' is invalid.
 Try again please


Hot Beverage Menu

A: Coffee         $1.00
B: Tea            $ .75
C: Hot Chocolate  $1.25
D: Cappuccino     $2.50


Enter the beverage A,B,C, or D you desire
Enter E to exit the program

e
 Please come again
	\end{verbatim}\normalsize
	
	\textbf{Exercise 2:} In the previous section, I used \lstinline{'g'} and \lstinline{'G'} as example selections not in the valid range. The program returned my statement \texttt{Selection 'G' is invalid. Try again please} and prompted the menu once again.
	
	\textbf{Exercise 3:} Changing \lstinline{if (validBeverage == true)} to \lstinline{if (validBeverage)} yields no change in the function of the program, because in either case of \lstinline{validBeverage} being \lstinline{true} or \lstinline{false}, the evaluation of the contents of \lstinline{validBeverage} gives the same result as the evaluation of the expression \lstinline{(validBeverage == true)}.
	

\section*{\textsc{Lab 5.3} --- Working with the \texttt{for} Loop}
	
	\textbf{Exercise 1:} The \lstinline{static_cast} is necessary because the divisors are both \lstinline{int} variables, but the desired mean needs to use floating-point division. Removing the \lstinline{static_cast} truncates the answer to mean at the decimal point, rounding down to the closest integer. It has no effect on odd inputs.
	
	\textbf{Exercise 2:} Inputting a fractional number makes the program truncate the input to before the decimal place. For example, inputting \texttt{2.99} gives the result \texttt{The mean average of the first 2 positive integers is 1.5}, the same result you would get from an input of \texttt{2}.
	
	\textbf{Exercise 3:} The changes have been made to accomodate a specific range of positive numbers and are included in the file \texttt{for.cpp}. An alternate, overhauled version of the solution to the problem is included in the file \texttt{for.better.cpp}.
	

\section*{\textsc{Lab 5.4} --- Nested Loops}
\noindent	The appropriate modifications have been made to the file, which is included as \texttt{nested.cpp}.
	

\section*{\textsc{Lab 5.5} --- Reading and Writing to a File}
\noindent	The appropriate additions have been made to the file, which is included as \texttt{billfile.cpp}.


\section*{\textsc{Lab 5.6} --- Student Generated Code Assignments}

	The option 1 assignment is completed and included as \texttt{5.6.survey.cpp}.
	
	The option 2 assignment is completed and included as \texttt{5.6.melon.cpp}.


\end{document}
