% Find-the-Error Problems for Chapter 7
% written by Walter B. Vaughan

\documentclass[11pt]{article}
\author{Walter B. Vaughan\\
        \small CSC 134 -- Section 200 -- Fall 2014\\
        \small Catawba Valley Community College}
\title{Chapter 7 Find the Error \\
       \footnotesize{80, 82, 83, 87}}
\date{\vspace{-5ex}}

\usepackage{listings}
\usepackage[usenames,dvipsnames]{color}
\usepackage[margin=.5in]{geometry}
\usepackage{indentfirst}

\lstset{frame=tb,
    language=C++,
    columns=flexible,
    basicstyle={\small\ttfamily},
    numbers=left,
    numberstyle=\color{Gray},
    keywordstyle=\color{BlueViolet},
    commentstyle=\color{Gray},
    stringstyle=\color{OliveGreen},
    breaklines=true,
    breakatwhitespace=true,
    showstringspaces=false,
    tabsize=4
}

\begin{document}

\maketitle

\begin{description}
    \item[80.] \lstinline{size} is not initialized when it is given as the array length parameter, and it also needs to be declared as a \lstinline{const int}, not just an \lstinline{int}.
    \item[82.] The \lstinline{for} loop is run outside the bounds of the array \lstinline{table}. \lstinline{x < 20;} should be \lstinline{x < 10;}.
    \item[83.] The initialized values of \lstinline{hours} should be inside a set of braces. It should look like: \lstinline[mathescape]{$\{$ 8, 12, 16 $\}$;}
    \item[87.] You cannot assign an array to another array. Array variables contain pointers to the address of the first element, and C++ does not allow those addresses to be overwritten. One could use \lstinline{memcpy} since the array only contains \lstinline{int}s.
\end{description}

\end{document}
