% Checkpoint Exercises for Chapter 13 Homework Assignment
% written by Walter B. Vaughan

\documentclass[11pt]{article}
\author{Walter B. Vaughan\\
        \small CSC 134 -- Section 200 -- Fall 2014\\
        \small Catawba Valley Community College}
\title{Chapter 13 Checkpoint Exercises \\
       \footnotesize{13.1 - 13.12, 13.14, 13.7, 13.21, 31.22}}
\date{\vspace{-5ex}}
% Section 13.3, Defining an Instance of a Class: 13.1, 13.2, 13.3, 13.4, 13.5
% Section 13.6, Inline Member Functions: 13.6, 13.7, 13.8, 13.9, 13.10, 13.11
% Section 13.9, Destructors: 13.12, 13.14, 13.17
% Section 13.12, Arrays of Objects: 13.21, 13.22

\usepackage{listings}
\usepackage[usenames,dvipsnames]{color}
\usepackage[margin=.5in]{geometry}
\usepackage{indentfirst}

\lstset{frame=tb,
    language=C++,
    columns=flexible,
    keepspaces=true,
    basicstyle={\small\ttfamily},
    numbers=left,
    numberstyle=\color{Gray},
    keywordstyle=\color{BlueViolet},
    commentstyle=\color{Gray},
    stringstyle=\color{OliveGreen},
    breaklines=true,
    breakatwhitespace=true,
    showstringspaces=false,
    tabsize=4
}

\begin{document}

\maketitle


\section*{Section 13.3 --- Defining an Instance of a Class}
\begin{description}

    \item[13.1] False. Private members of a class can be defined after the public members as long as they are separated in the appropriate sections by the \lstinline{public} and \lstinline{private} labels.
    \item[13.2] B
    \item[13.3] A
    \item[13.4] C
    \item[13.5] The completed class definition, built from the skeleton provided and as instructed in the problem, is attached as \texttt{13.5.date.cpp} and looks like this:
\begin{lstlisting}
class Date
{
    private:
        unsigned month;
        unsigned day;
        unsigned year;
    public:
        // alternative constructor, uses explicit number input
        Date(unsigned m, unsigned d, unsigned y) {
            year = y; month = m; day = d;
        }
        // string-based input constructor
        Date(std::string text) {
            std::istringstream s (text);
            char slash; // junk character to discard slashes
            s >> month >> slash >> day >> slash >> year;
        }
        std::string get() {
            std::ostringstream s;
            s << month << "/" << day << "/" << year;
            return s.str();
        }
}
\end{lstlisting}
\end{description}

\section*{Section 13.6 --- Inline Member Functions}
\begin{description}
    \item[13.6] Declaring a class's member variables as \lstinline{private} allows an object to force specific behavior of its data. For example, this could allow a class to override values given to it if they were deemed inappropriate, such as a negative value for the length of a line.
    \item[13.7] Code outside of a class can store, retrieve, or otherwise modify private members of a class through member accessors and mutators or through other public member functions.
    \item[13.8] A class specification file contains class declarations or prototypes, whereas a separate class implementation file would contain the actual definitions matching the declarations of a specification file.
    \item[13.9] An include guard prevents a header file from being \lstinline{#include}ed more than once by setting a macro constant unique to the header file which is originally undefined. Once defined by a preprocessor directive, the contents of the specification file will not be re-introduced to the compiler.
    \item[13.10] This is a table showing in what file components of a class should be placed. \\ \vspace*{1mm} \\
        \begin{tabular}{r || l}
            Class Component & Stored In \\
            \hline
            \lstinline{BasePay} class declaration & Specification file \\
            \lstinline{BasePay} member function definitions & Implementation file \\
            \lstinline{Overtime} class declaration & Specification file \\
            \lstinline{Overtime} member function definitions & Implementation file \\
        \end{tabular}
    \item[13.11] An inline member function is a member function that is defined (not just declared) inside of the class declaration.
\end{description}

\section*{Section 13.9 --- Destructors}
\begin{description}
    \item[13.12] The purpose of a constructor is to initialize an object's attributes.
    \item[13.14] A
    \item[13.17] A
\end{description}

\section*{Section 13.12 --- Arrays of Objects}
\begin{description}
    \item[13.21] The program will display the following:
\begin{verbatim}
10
20
50
\end{verbatim}
    \item[13.22] The program will display the following:
\begin{verbatim}
4
2
2
4
\end{verbatim}
\end{description}

\end{document}
